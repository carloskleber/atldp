% !TeX encoding = UTF-8
% !TeX spellcheck = en_US
\documentclass[]{article}
\usepackage[utf8]{inputenc}
\usepackage[T1]{fontenc}
\usepackage{amsmath,amsfonts,amssymb}
\usepackage{tabulary}
\usepackage{graphicx,booktabs,caption,xcolor,listings,listings,mathtools,nicefrac,rotating,threeparttable}

\title{The sag-tension problem}
\author{Carlos Kleber c. Arruda}

\begin{document}

\maketitle

\begin{abstract}
Just sketching the problem in some pretty equations...
\end{abstract}

\section{Model}

Premises:

Span depth ratio > 1/8, or inclined supports: the exact solution must be used

Span depth ratio < 1/8 and aligned supports: parabolic approximation can be used

Approaches

* Stationary
* Eigenfrequency (Vibration)
* Time dependent

Equação do cabo para vão nivelado
%
\begin{align}
	\frac{\partial}{\partial s} \left( T \frac{dx}{ds}\right) & = 0 \\
	\frac{\partial}{\partial s} \left( T \frac{dy}{ds}\right) & = -mg
\end{align}

Para pequenas flechas, aproxima-se pela parábola
%
\begin{equation}
	y = \frac{m g l^2}{2 H} \left[ \frac{x}{l} - \left( \frac{x}{l} \right)^2 \right]
\end{equation}


\begin{align}
	\frac{\partial}{\partial s} \left[ (T + \tau) \left( \frac{dx}{ds} + \frac{\partial u}{\partial s}\right) \right] & = \rho A \frac{\partial^2 u}{\partial t^2} \\
	\frac{\partial}{\partial s} \left[ (T + \tau) \left( \frac{dy}{ds} + \frac{\partial v}{\partial s}\right)\right] & = \rho A \frac{\partial^2 v}{\partial t^2} -\rho A g \\
	\frac{\partial}{\partial s} \left[ (T + \tau) \frac{\partial w}{\partial s} \right] & = \rho A \frac{\partial^2 w}{\partial t^2}	
\end{align}


\begin{equation}
	\ddot{q} + \mu \dot{q} + q + c_2 q^2 + c_3 q^3 = f(t)
\end{equation}

Sendo $q$ a coordenada modal, $f$ o vetor arbitrário de força externa, 

Reduced order model (ROM).



\end{document}
